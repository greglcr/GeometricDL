\documentclass[sigconf]{acmart}
\setcopyright{none} % Disable copyright
\acmDOI{} % Remove DOI

\settopmatter{printacmref=false} % Removes ACM reference format
\renewcommand\footnotetextcopyrightpermission[1]{} % Removes the footer

\begin{document}
\title{Fake News Detection Report}


\author{Clément Rouvroy}
\authornote{Both authors contributed equally to this research.}
\email{clement.rouvroy@ens.psl.eu}
\affiliation{%
  \institution{Ecole normale supérieure}
  \city{Paris}
  \state{Île-de-France}
  \country{FR}
}

\author{Grégoire Le Corre}
\authornotemark[1]
\email{gregoire.le.corre@ens.psl.eu}
\affiliation{%
  \institution{Ecole normale supérieure}
  \city{Paris}
  \state{Île-de-France}
  \country{FR}
}


\author{Niloo S?}
\authornotemark[1]
\email{ns2434@caa.columbia.edu}
\affiliation{%
  \institution{?}
  \city{?}
  \state{?}
  \country{?}
}
\renewcommand{\shortauthors}{Le Corre, Rouvroy, S?}

\begin{abstract}
  Na,g
\end{abstract}

%%
%% The code below is generated by the tool at http://dl.acm.org/ccs.cfm.
%% Please copy and paste the code instead of the example below.
%%

%\begin{teaserfigure}
%  \includegraphics[width=\textwidth]{sampleteaser}
%  \caption{Seattle Mariners at Spring Training, 2010.}
%  \Description{Enjoying the baseball game from the third-base
%  seats. Ichiro Suzuki preparing to bat.}
%  \label{fig:teaser}
%\end{teaserfigure}



\maketitle

\section{Dataset}

\subsection{Original dataset}
Given a URL $u$, they build a graph $G_u$ on tweets like this:
\begin{itemize}
  \item $T_u$ is the set of tweets related to the URL. A tweet is related to the URL if it mention it.
  \item There is an edge between two nodes (tweet) $i,j$ if $i$ follows $j$, or $j$ follows $i$, or the news spread from $i$ to $j$, or the news spread from $j$ to $i$. By \emph{news spreading} they mean that $i$ is retweeting $j$.
\end{itemize}

On this graph, they define \emph{spreading trees} by sorting the nodes by timestamps. The first node (the earlier one) is the root and then, given a node (tweet) $t_u^i$, if he follows an author of a tweet in $\{t_u^0,\dots,t_u^{i-1}\}$ then there is an edge between the very last tweet in $\{t_u^0,\dots,t_u^{i-1}\}$ that has been written by an author the author of $t_u^i$ is following. If he follows no one then the edge is between the tweet from the most influent user (the most recent from it) and $t_u^i$.

Hence here they have access to two distinct objects: the graph $G_u$ and the spreading trees. Likely for us (otherwise this would be impossible to replicate), there model is learning on the spreading trees, called \emph{cascades}. Moreover, the URL information is only used to classify the cascades between the fakes and the reals. They have around $10^5$ cascades in total. 

Each node is composed of a feature vector that encodes user profile information, content of the tweet and network and spreading information (social connections between the users, number of followers and friends, ... \emph{i don't understand how they manage to do this, if someone know I would like, are they using transformer ?})
\subsection{Our dataset}

Our dataset is GNN-FakeNews and it was extracted using the FakeNetTools, that requires an X's api key to generate dataset (which is limited to a hundred pull per month so not really usable for us). This dataset consists of roughly 6000 cascades so is less important, but is built the same way. The only difference is that a node has a feature that is the concatenation of a vector representing its profile informations and a vector that has been created using BERT on the content of the tweet.

\section{Model}

This is the same as the paper Niloo, see page 6, figure 5, we even took the Hinge Loss and the SELU activation function.

\end{document}
\endinput